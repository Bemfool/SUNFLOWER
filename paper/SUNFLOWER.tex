
\documentclass[a4paper, 10pt, conference]{ieeeconf}   
\usepackage[utf8]{inputenc}
\usepackage{dtk-logos} % for BibTeX stylized logo 
\overrideIEEEmargins


\title{\LARGE \bf
The review of Automated Machine learning
}

\author{He Xin$^{1}$ and Wang Zhichun$^{2}$
}

\begin{document}

\maketitle
%\thispagestyle{empty}
%\pagestyle{empty}

%%%%%%%%%%%%%%%%%%%%%%%%%%%%%%%%%%%%%%%%%%%%%%%%%%%%%%%%%%%%%%%%%%%%%%%%%%%%%%%%
\begin{abstract}

Test test testTest test testTest test testTest test test

Test test testTest test testTest test testTest test test


\end{abstract}

\section{INTRODUCTION}

As we all know(\cite{xie_genetic_2017}), deep learning, which has been used in a lot of research fields including image classification, image recognition, machine translation, has achieved remarkable achievements in those tasks. Take the image classification as an example, AlexNet () outperformed traditional computer vision methods on ImageNet (Russakovsky et al., 2015), which was in turn outperformed by VGG nets (Simonyan \& Zisserman, 2015), then ResNets (He et al., 2016) etc.


\section{METHODS}

As we all know(\cite{xie_genetic_2017}), deep learning, which has been used in a lot of research fields including image classification, image recognition, machine translation, has achieved remarkable achievements in those tasks. Take the image classification as an example, AlexNet () outperformed traditional computer vision methods on ImageNet (Russakovsky et al., 2015), which was in turn outperformed by VGG nets (Simonyan \& Zisserman, 2015), then ResNets (He et al., 2016) etc.


\subsection{Bayesian Optimization}

Test test testTest test testTest test testTest test test
As we all know(\cite{xie_genetic_2017}), deep learning, which has been used in a lot of research fields including image classification, image recognition, machine translation, has achieved remarkable achievements in those tasks. Take the image classification as an example, AlexNet () outperformed traditional computer vision methods on ImageNet (Russakovsky et al., 2015), which was in turn outperformed by VGG nets (Simonyan \& Zisserman, 2015), then ResNets (He et al., 2016) etc.



\subsection{Gradient-based}

Test test testTest test testTest test testTest test test
As we all know(\cite{xie_genetic_2017}), deep learning, which has been used in a lot of research fields including image classification, image recognition, machine translation, has achieved remarkable achievements in those tasks. Take the image classification as an example, AlexNet () outperformed traditional computer vision methods on ImageNet (Russakovsky et al., 2015), which was in turn outperformed by VGG nets (Simonyan \& Zisserman, 2015), then ResNets (He et al., 2016) etc.




\subsection{Meta Learning}

Test test testTest test testTest test testTest test test

As we all know(\cite{xie_genetic_2017}), deep learning, which has been used in a lot of research fields including image classification, image recognition, machine translation, has achieved remarkable achievements in those tasks. Take the image classification as an example, AlexNet () outperformed traditional computer vision methods on ImageNet (Russakovsky et al., 2015), which was in turn outperformed by VGG nets (Simonyan \& Zisserman, 2015), then ResNets (He et al., 2016) etc.





\subsection{Evolutionary Algorithm}
Test test testTest test testTest test testTest test test


As we all know(\cite{xie_genetic_2017}), deep learning, which has been used in a lot of research fields including image classification, image recognition, machine translation, has achieved remarkable achievements in those tasks. Take the image classification as an example, AlexNet () outperformed traditional computer vision methods on ImageNet (Russakovsky et al., 2015), which was in turn outperformed by VGG nets (Simonyan \& Zisserman, 2015), then ResNets (He et al., 2016) etc.




\subsection{Reinforcement Learning}


Test test testTest test testTest test testTest test test

Test test testTest test testTest test testTest test test

Test test testTest test testTest test testTest test test



\section{Comparison and Analysis}

Test test testTest test testTest test testTest test test
Test test testTest test testTest test testTest test test

Test test testTest test testTest test testTest test test

\subsection{Units}


Test test testTest test testTest test testTest test test
Test test testTest test testTest test testTest test test

Test test testTest test testTest test testTest test test


\begin{itemize}

\item Test test test
\item Test test test

\end{itemize}



\section{CONCLUSIONS}


Test test testTest test testTest test testTest test test

Test test testTest test testTest test testTest test test

Test test testTest test testTest test testTest test test

\addtolength{\textheight}{-12cm}   % This command serves to balance the column lengths
                                  % on the last page of the document manually. It shortens
                                  % the textheight of the last page by a suitable amount.
                                  % This command does not take effect until the next page
                                  % so it should come on the page before the last. Make
                                  % sure that you do not shorten the textheight too much.

%%%%%%%%%%%%%%%%%%%%%%%%%%%%%%%%%%%%%%%%%%%%%%%%%%%%%%%%%%%%%%%%%%%%%%%%%%%%%%%%



%%%%%%%%%%%%%%%%%%%%%%%%%%%%%%%%%%%%%%%%%%%%%%%%%%%%%%%%%%%%%%%%%%%%%%%%%%%%%%%%



%%%%%%%%%%%%%%%%%%%%%%%%%%%%%%%%%%%%%%%%%%%%%%%%%%%%%%%%%%%%%%%%%%%%%%%%%%%%%%%%
\section*{APPENDIX}

Test test
Test test testTest test testTest test testTest test test

Test test testTest test testTest test testTest test test

\section*{ACKNOWLEDGMENT}

Test test testTest test testTest test testTest test test

Test test testTest test testTest test testTest test test
Test test

%%%%%%%%%%%%%%%%%%%%%%%%%%%%%%%%%%%%%%%%%%%%%%%%%%%%%%%%%%%%%%%%%%%%%%%%%%%%%%%%


\nocite{*}
\bibliographystyle{ieeetran}
\bibliography{SUNFLOWER}


\end{document}
